\chapter{Discussion}
In this chapter the work done in conjunction to this thesis will be discussed. 
First I will talk about the pros and cons using \gls{IBC} in \gls{NDN}.
Then I will discuss the \gls{HSS} and possible drawbacks in the system. 
Scalability issues and other applicable networks for the application will be mentioned.

\section{Identity-Based Cryptography in Named Data Networking}
Concerning key revocation, one suggestion has been to add a monthly timestamp to the \gls{name}, but the the \gls{PKG} has to renew private keys for everybody each month. 
This solution do not scale very well due to a lot of computation at the \gls{PKG}.
With the \gls{FSM}, every user will be notified when a identity is revoked.
There is no use for periodically checking names.
But the renewal of keys might not be an issue in the \gls{HSS}. \todo{more..}

\todo{refactor this section}
Can authenticate \gls{data} even using insecure DNS or HTTP. 
There is only one linkage between the \gls{name} and the content, and if the user obtains the right \gls{MPK}, there is no doubt where the \gls{data} originates from and that it is not altered.
In RSA public key cryptography we have to find the key related to the signature. 
In worst case this will be equivalent of retrieving the \gls{MPK} each time, which is not likely. 
Or the \gls{MPK} can be appended to the message.

usability 



\section{Development Usability in Named Data Networking}
Once a basic perception of the \gls{NDN} architecture is understood, it is easy to begin developing.
The \gls{PyNDN2} framework comes with good examples of how to develop simple applications with packets that are signed and encrypted.

The concept of naming \gls{data} introduces more simplicity, but also a new way of application design thinking.
Addressing is dealt with one place in the architecture compared to an equivalent system over \gls{IP}. 
Security is easily applied in \gls{NDN}.

\section{Health Sensor System}
The application is not tested on with real sensors, hence I cannot conclude with anything regarding the computational power of such devices, nor the life time of the battery when performing \gls{IBE}.  
\todo{more on this}

In~\autoref{ibc-performance} we can see that \gls{IBC} is performing better than regular asymmetric cryptography, RSA. 

Encrypting with same symmetric key for each set of \gls{data}, limits the encryption computation for the device if several devices requests the same \gls{data}.
Using a unique key for each time \gls{data} is requested is more secure and can be used for more sensitive content.

Who can play the role of a device?
As mentioned in~\autoref{rendezvous_authentication} and~\autoref{init} there should be a limitation of which devices that can be initialized to the trust domain.
\todo{hm... something here?}

Storing the \gls{SK} in a secure fashion.

Preloading secret key, offline mode, yet the initialization protocol should be used to do this, unless the sharing is done in a wired environment. 
\gls{NFC} signal is hard (impossible?) to eavesdrop, thus the initialization protocol is not needed. 


\section{Scalability}
Distributing the \gls{ID}-list can be an issue, as the list can grow linearly with the number of participants in the trust domain.
However, this might not be a huge problem in the use cases that is addressed in this thesis, considering that the number of devices in the \gls{HSS} will not grow larger than e.g. 100 devices. 

\section{Sync}
The sync application makes it possible for users to know who has a valid public key within the \gls{PKG}s domain.
One drawback with the key distribution using \gls{FSM} is that for the sender to be 100\% sure that the message is encrypted with the latest \gls{ID}, the sender has to rely on that it has received the latest sync state available from the \gls{PKG}.
Likewise when a \gls{receiver} verifies a signature, it has to rely on the same principle to be able to know if the belonging \gls{ID} is still valid.
Since the scalability is not that big of an issue in this scenario, the monthly (or even more often) timestamp appended to the \gls{ID} might be a good solution to reduce the time of exposure when compromised.

In~\cite{DBLP:conf/spw/StajanoA99} Frank Stajano and Ross Anderson mentions possible \gls{DoS} attacks, such as radio jamming and battery exhaustion. 
All applications that relies on some sort of crucial information derived using \gls{FSM} (\autoref{file-sync}) are vulnerable to this kind of \gls{DoS}.

\section{Other Use Cases}
The trust model used in the \gls{HSS} can be used in any network where the use of a \gls{TTP} is accepted. 
Such a system can for instance be:
\begin{enumerate}
	\item Home Automation Systems
	\item \gls{BAS}
	\item \gls{BMS}
	\item Open mHealth
\end{enumerate}


