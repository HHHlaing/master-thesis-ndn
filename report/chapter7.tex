\chapter{Conclusion and Future Work}\label{chp7:conclusion}
In this chapter the conclusion of this thesis will be presented and the future work will be listed.

\section{Conclusion}
In this thesis I have developed a Health Sensor System in Python with \gls{IBC} used for signing and verification, encryption and decryption.
The system is built over the new network protocol called Named Data Networking.
I have suggested how a secure system easily can be implemented, achieving confidentiality, integrity and authenticity, as well as trust.

The system is tested to see how the suggested protocols for device registration and data pull performs with \gls{IBC}. 
This work is an attempt to show how applicable \gls{NDN} together with \gls{IBC} are for \gls{IoT}, and to design secure protocols for local device networks.

\gls{NDN} facilitates a lot of concepts that shows to be a huge benefit for todays Internet, and the predicted increase of \gls{IoT}.
The naming of content and content routing provides usability to \gls{IoT} and \gls{wsn}.
The concept of \gls{IBC} appears to be highly applicable to \gls{IoT} and \gls{wsn}~\cite{Patil:2012:SWS:2464778}.
Running \gls{IBC} over \gls{NDN}, makes it even more practical, because of the naming of content concept that \gls{NDN} is built upon. 
As mentioned it is easier to secure data, relate data to publisher, and authenticate that the publisher is aware of what content it published. 
I believe that using \gls{IBC} in a \gls{wsn} running over \gls{NDN} should make applications with security less complex and more practical than using security such as RSA. 

Developing applications on top of \gls{NDN} is easy once a basic perception of the \gls{NDN} architecture is understood.
The \gls{PyNDN2} framework comes with good examples of how to develop simple applications with packets that are signed and encrypted.

\section{Future Work}
The implementation of the \gls{HSS} does not include integration of the \gls{FSM}, which is a part of the future work.
The system is not tested on relevant sensors and devices to measure latency and performance.

An implementation of a full worthy \gls{IBC} solution in \gls{PyNDN2} is not implemented.
This implementation should include making \gls{IBC} as a part of the PyNDN2 framework, so that developers easily can make use of \gls{IBE} and \gls{IBS} performing encryption, decryption, signing and verification.

The \gls{IBC} schemes used in the Charm framework does not provide a scheme that implements \gls{IBE} and \gls{IBS} together in one scheme.
This should not be a huge task to implement, but it will decrease the device registration round-trip time as well as minimizing the use of several keys, i.e. easier key management.
It is explained how to derive a signature scheme from any \gls{IBE} scheme~\cite[Section 4]{DBLP:conf/crypto/Waters09}.
