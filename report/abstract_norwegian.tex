\pagestyle{empty}
\begin{abstract}

IP nettverket ble bygd for flere ti\r{a}r siden, og med dagens bruk av Internet ser vi at en ny nettverksprotokoll er s\r{a}rt trengt.
Named Data Networking (NDN) er en foresl\r{a}tt nettverksprotokoll som baserer seg p\r{a} innhold, istedenfor punkt-til-punkt arkitekturen som er grunnlaget for IP.
Selv med flere ti\r{a}rs bruk av Internet, er enn\r{a} ikke problemene med Public Key Infrastructure (PKI) l\o{}st. 
I NDN, har man heller ikke klart \r{a} finne en l\o{}sning p\r{a} dette.
Identitetsbasert kryptografi (IBC) viser seg \r{a} være anvendelig til tr\r{a}dl\o{}se sensornettverk, og enda mer n\r{a}r sensornettverket kommuniserer over NDN.

I denne masteroppgaven forklarer jeg NDN arkitekturen og de grunnlegende prinsippene i IBC.
Jeg modellerer og implementerer en applikasjon for \r{a} demonstrere bruken av IBC over NDN i et tenkt sensornettverk.

Implementasjonen og testingen av mitt bidrag verifiserer relevansen av IBC i et sensornettverk som kj\o{}rer over NDN, samt brukervennligheten rundt det \r{a} utvikle applikasjoner over NDN.

Jeg formelt og uformelt beviser sikkerheten i protokollene som er foresl\r{a}tt til enhetsregistrering og data foresp\o{}rsel i applikasjonen.

\end{abstract}