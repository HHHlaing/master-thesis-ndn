\chapter{Key Infrastructure}
This chapter will present the concept of \gls{PKI} and \gls{IBE}, as well as problems and possible solutions. 

\section{Public Key Infrastructure}

X.509 (i.e. \gls{PKI}). Issues: certificates --> distribution and revocation.

\section{Identity-Based Cryptography}\label{ibc}
\gls{IBE} was first proposed by Shamir~\cite{DBLP:conf/crypto/Shamir84} in 1984. 
The concept of \gls{IBE} builds upon every user having an \gls{ID} that is used as the public key. 
This \gls{ID} can be anything, i.e. email, phone number, \gls{SSN}, or a Name (~\autoref{name}).
This eliminates the need of certificates.
Shamir did propose a scheme for \gls{IBS}, but not a scheme for \gls{IBE}. 
The \gls{IBE} implementation remained unsolved until 2001, when Dan Boneh and Matthew K. Franklin proposed~\cite{DBLP:conf/crypto/BonehF01}.
However the scheme has only been shown to be secure with a random oracles model~\cite{DBLP:journals/iacr/Waters04}, hence less practical.


\gls{IBE} is based upon performing cryptography with a publicly know \gls{ID}.
Since the \gls{ID} can be practically anything it is highly applicable for \gls{NDN} where the \gls{ID} can be a Name (``/ndn/no/ntnu/haakon").
Hence the Name becomes the public key. 

There is a \gls{TTP} in \gls{IBE} that is called \gls{PKG}.
The \gls{PKG}s task is to produce a private key that corresponds to a given ID and provide 

\begin{enumerate}
  \item Setup generates a key pair, \gls{MPK} and \gls{MSK}. These keys are used to extracting private keys, encryption and decryption.
  \item Extract generates a private key from a given ID. 
  \item Encrypt(\gls{MPK}, ID, message)
  \item Decrypt(\gls{MPK}, private key, cipher)
  \item Signing(\gls{MPK}, private key, message)
  \item Verify(\gls{MPK}, ID, message, signature)
\end{enumerate}

\todo{figure of PKG and nodes}

Thoroughness of the name allocation \gls{NRS} 
Identity-based cryptography~\cite{DBLP:conf/icnp/ZhangCXWSW11} in \gls{NDN}

Key Revocation in IBE ~\cite{DBLP:journals/iacr/BoldyrevaGK12} 

To encrypt a message with \gls{IBE}, the user encrypts a \gls{CEK} with the recipents \gls{IBE} public key.
Then the user encrypts the message with the \gls{CEK}~\cite[section 2.2.2]{rfc5408}

Drawbacks
If \gls{PKG} is compromised. Adversary has private key to all nodes that used the compromised \gls{PKG}
\gls{PKG} can read and write messages related to the node, because it has all private keys, i.e. \gls{MITM}.
\gls{PKG} and the requesting node has to establish a secure channel. 


\section{Key Distribution}
Instead of in \gls{PKI}, where each public key is signed by a certificate authority and the generated certificate is sent as a response in \gls{HTTPS}, then validated by the the client, we want to make the certificate authority obsolete by distributing every public key and trust the name (e.g. ``/ntnu"). 
Name sync (same application as Name Sync).

\section{Key Revocation}
Key Revocation becomes obsolete with Name Sync. 
