\chapter{Key Infrastructure}
This chapter will present the concept of \gls{PKI} and \gls{IBE}, as well as problems and possible solutions. 

\section{Public Key Infrastructure}

X.509 (i.e. \gls{PKI}). Issues: ... ... .. . . ..

\section{Identity-Based Cryptography}\label{ibc}
\gls{IBE} was first presented by Shamir~\cite{DBLP:conf/crypto/Shamir84} in 1984. 
Shamir did propose a scheme for \gls{IBS}, but it was not until recently a scheme of the concept was implemented. 
The \gls{IBE} implementation remained unsolved until 2001, when Dan Boneh and Matthew K. Franklin proposed~\cite{DBLP:conf/crypto/BonehF01}.

\gls{IBE} is based upon performing cryptography with a publicly know ID.
The ID, can be practically anything.
This makes it highly applicable for \gls{NDN} where the ID can be a Name (/ndn/no/ntnu/haakon).
Hence the Name becomes the public key. 

There is a \gls{TTP} in \gls{IBE} that is called \gls{PKG}.
The \gls{PKG}s task is to produce a private key that corresponds to a given ID and provide 

\begin{enumerate}
  \item Setup generates a key pair, \gls{MPK} and \gls{MSK}. These keys are used to extracting private keys, encryption and decryption.
  \item Extract generates a private key from a given ID. 
  \item Encrypt(\gls{MPK}, ID, message)
  \item Decrypt(\gls{MPK}, private key, cipher)
  \item Signing(\gls{MPK}, private key, message)
  \item Verify(\gls{MPK}, ID, message, signature)
\end{enumerate}

\todo{figure of PKG and nodes}

Thoroughness of the name allocation \gls{NRS} 
Identity-based cryptography~\cite{DBLP:conf/icnp/ZhangCXWSW11} in \gls{NDN}

Key Revocation in IBE ~\cite{DBLP:journals/iacr/BoldyrevaGK12} 

\gls{CEK} is decryptes with the private key
RFC~\cite{rfc5408}

Drawbacks
If \gls{PKG} is compromised. Adversary has private key to all nodes that used the compromised \gls{PKG}
\gls{PKG} can read and write messages related to the node, because it has all private keys, i.e. \gls{MITM}.
\gls{PKG} and the requesting node has to establish a secure channel. 

\subsection{..}
Name sync (same application as Public Key Sync).


\section{Key Distribution}
Instead of in \gls{PKI}, where each pk is signed by a certificate authority and the generated certificate is sent as a response in \gls{HTTPS}, then validated by the the client, we want to make the certificate authority obsolete by distributing every PK and rather trust the name (e.g. /ntnu/). 

\section{Key Revocation}
Key Revocation becomes obsolete with Public Key Sync. 
