\chapter{Key Infrastructure}

\section{Public Key Infrastructure}
\gls{PKI}

\section{Identity-Based Cryptography}
\gls{IBE} presented by Shamir~\cite{DBLP:conf/crypto/Shamir84} in 1984.

Thoroughness of the name allocation \gls{NRS} 
Identity-based cryptography~\cite{DBLP:conf/icnp/ZhangCXWSW11} in \gls{NDN}

Key Revocation in IBE ~\cite{DBLP:journals/iacr/BoldyrevaGK12}

Drawbacks
If \gls{PKG} is compromised. Adversary has private key to all nodes that used the compromised \gls{PKG}
\gls{PKG} can read and write messages related to the node, because it has all private keys, i.e. \gls{MITM}.
\gls{PKG} and the requesting node has to establish a secure channel. 

\subsection{..}
Name sync (same application as Public Key Sync).


\section{Key Distribution}
Instead of in \gls{PKI}, where each pk is signed by a certificate authority and the generated certificate is sent as a response in \gls{HTTPS}, then validated by the the client, we want to make the certificate authority obsolete by distributing every PK and rather trust the name (e.g. /ntnu/). 

\section{Key Revocation}
Key Revocation becomes obsolete with Public Key Sync. 
