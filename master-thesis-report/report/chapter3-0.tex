\chapter{File Synchronization over Named Data Network}\label{chp3:application}
This chapter will present the work done in conjunction with this thesis. 
Explaining what the synchronization application is built upon and its purpose over the \gls{NDN} network. 

\section{ChronoSync}
A distributed (server-less) synchronization application developed by the \gls{NDN} team. 

ChronoSync~\cite{DBLP:conf/icnp/ZhuA13}
Encode data into crypto digest, i.e. a state digest, to exchange states between all parties. 
If the state is equal to the one stored locally, then nothing should happen.
If not, two options are available; 
- the differences of the dataset state can be directly inferred
and sent as the response to the sync interest if the state
digest is the same as one of the previous local state
digests;
- a state reconciliation method is used to determine the differences of the knowledge if the state digest is unknown
(for example, when recovering from a network partition).


Sync interest
Sync data
Recovery interest
Recovery data


\section{Multicast By Nature}
\todo{figure of multicast}


\section{FileSync.py application}
~\autoref{apx:file-sync-code}
FileSync is a python application that will synchronize all files in a specified path, with all participants within the synchronization room.


\subsection{ChronoSync}

\subsection{File Watcher}
A library that makes it possible to watch files in an \gls{OS} is WatchDog\todo{https://pypi.python.org/pypi/watchdog}.
This library 
